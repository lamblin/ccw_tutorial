%==============================================================================
% PREAMBLE
%==============================================================================
\documentclass[mathserif, xcolor=dvipsnames]{beamer}
% \usepackage{helvet}
\usepackage[T1]{fontenc}
\usepackage{amsmath, amsfonts,amsthm}
\usepackage{url}
\usepackage{listings}
\usepackage{color}

\usetheme[height=10mm]{Rochester}
\usecolortheme[named=MidnightBlue]{structure}
\setbeamertemplate{navigation symbols}{}

\title{CCW: Pylearn2}
\subtitle{Not your grandfather's machine learning library}
\author{Pascal Lamblin\\
slides by Vincent Dumoulin}
\date{October 2nd, 2014}

%==============================================================================
% BODY
%==============================================================================
\begin{document}

%------------------------------------------------------------------------------
% TITLE
%------------------------------------------------------------------------------
\begin{frame}[plain]
    \titlepage
\end{frame}

%------------------------------------------------------------------------------
% OBJECTIVES
%------------------------------------------------------------------------------
\begin{frame}
    \frametitle{Objectives}
    \begin{itemize}\addtolength{\itemsep}{0.5\baselineskip}
        \item{
            {\Large Manage an experiment using Pylearn2}
            \begin{itemize}
                \item{Anatomy of a \textbf{YAML} experiment file}
                \item{The \textbf{train.py} script}
            \end{itemize}
        }
        \item{
            {\Large High-level understanding of Pylearn2}
            \begin{itemize}
                \item{\textbf{Train} object}
                \item{\textbf{TrainingAlgorithm} object}
                \item{\textbf{Model} object}
                \item{\textbf{Dataset} object}
                \item{\textbf{Cost} object}
                \item{\textbf{Monitor} object}
                \item{\textbf{TerminationCriterion} object}
                \item{\textbf{TrainExtension} object}
                \item{\textbf{utils} module}
                \item{\textbf{scripts} directory}
            \end{itemize}
        }
        \item{
            {\Large Extend Pylearn2 to suit your needs}
        }
    \end{itemize}
\end{frame}

%------------------------------------------------------------------------------
% WHAT
%------------------------------------------------------------------------------
\begin{frame}
    \frametitle{What is Pylearn2?}
    \begin{itemize}\addtolength{\itemsep}{2.0\baselineskip}
        \item{\LARGE Machine learning \textbf{prototyping} library}
        \item{\LARGE Built on top of Theano}
        \item{\LARGE Easy to extend}
    \end{itemize}
\end{frame}

%------------------------------------------------------------------------------
% FOLLOW ALONG
%------------------------------------------------------------------------------
\begin{frame}
    \frametitle{Accompanying material}
    \begin{itemize}\addtolength{\itemsep}{3.0\baselineskip}
        \item{\Large Make sure you have access to a machine that has Pylearn2
              and its dependencies installed}
        \item{\Large The whole presentation and accompanying material can be
              found on Github here: \url{http://goo.gl/nuJw9R}}
    \end{itemize}
\end{frame}

%------------------------------------------------------------------------------
% CASE STUDY
%------------------------------------------------------------------------------
\begin{frame}[t]
    \frametitle{Case study: softmax regression on MNIST digits
                \footnote{Adapted from Ian Goodfellow's softmax regression
                          iPython Notebook tutorial (\url{http://goo.gl/qSdAjA})}}

    \begin{center}
    \begin{minipage}{0.05\textwidth}
    \begin{figure}[H]
        \raggedleft
        \includegraphics[width=\textwidth]{mnist_4.png}
    \end{figure}
    \end{minipage}
    \begin{minipage}{0.9\textwidth}
        \raggedright
        $\Rightarrow ([0., 0., 0., \ldots, 0., 0.],
                      [4])$
    \end{minipage}
    \end{center}

    \begin{itemize}\addtolength{\itemsep}{2.0\baselineskip}
        \item{Predict the class: learn $p(\mathbf{y} \mid \mathbf{x})$}
        \item{$\mathbf{x} \in [0, 1]^{784}$ ($28 \times 28$ pixels unrolled
              into a 784-dimension vector)}
        \item{$\mathbf{y} \in \{0, 1, \ldots, 9\}$}
    \end{itemize}
\end{frame}

%------------------------------------------------------------------------------
% CASE STUDY
%------------------------------------------------------------------------------
\begin{frame}[t]
    \frametitle{Case study: softmax regression on MNIST digits}

    \begin{center}
    \begin{minipage}{0.05\textwidth}
    \begin{figure}[H]
        \raggedleft
        \includegraphics[width=\textwidth]{mnist_4.png}
    \end{figure}
    \end{minipage}
    \begin{minipage}{0.9\textwidth}
        \raggedright
        $\Rightarrow ([0., 0., 0., \ldots, 0., 0.],
                      [4])$
    \end{minipage}
    \end{center}

    \begin{itemize}\addtolength{\itemsep}{0.5\baselineskip}
        \item{
            Model $p(\mathbf{y} \mid \mathbf{x})$ as
            \begin{equation*}
                p(\mathbf{y} \mid \mathbf{x})
                = \text{softmax}(\mathbf{x}; \mathbf{W}, \mathbf{b})
                = \frac{\exp(\mathbf{x}^T \mathbf{W} + \mathbf{b})}
                       {\sum_i \exp(\mathbf{x}^T \mathbf{W} + \mathbf{b})_i}
            \end{equation*}
            with $\mathbf{W}$ a $784 \times 10$ matrix and $\mathbf{b}$ a
            10-dimension vector
        }
        \item{
            Measure performance using negative log-likelihood (NLL):
            \begin{equation*}
                \mathcal{L}(\mathcal{D}, \mathbf{W}, \mathbf{b})
                = -\sum_{\mathbf{x}, \mathbf{y} \in \mathcal{D}}
                    \log p(\mathbf{y} \mid \mathbf{x})
            \end{equation*}
        }
        \item{
            Train by stochastic gradient descent:
            \begin{equation*}
                \theta \gets \theta - \eta \nabla \mathcal{L}
            \end{equation*}
        }
    \end{itemize}
\end{frame}

%------------------------------------------------------------------------------
% HOW TO LAUNCH
%------------------------------------------------------------------------------
\begin{frame}[fragile]
\frametitle{Launch an experiment: how-to}
\begin{center}
{\Huge Launch}

\vspace{1cm}

\lstset{escapechar=@, language=bash}
\begin{lstlisting}
$ python ${PYLEARN2_LOCATION}/scripts/train.py \
> softmax_regression.yaml
\end{lstlisting}

\vspace{1cm}

{\Huge What happened?}
\end{center}

\end{frame}

%------------------------------------------------------------------------------
% TRAIN.PY
%------------------------------------------------------------------------------
\begin{frame}
    \frametitle{Launch an experiment: \textbf{train.py} script}
    \Large
    \begin{itemize}\addtolength{\itemsep}{2.0\baselineskip}
        \item{Takes a YAML file as argument}
        \item{Instantiates the object(s) listed in the file}
        \item{Calls its (their) \textbf{main\_loop} method}
    \end{itemize}

\end{frame}

%------------------------------------------------------------------------------
% ANATOMY
%------------------------------------------------------------------------------
\begin{frame}[fragile]
\frametitle{Launch an experiment: \textbf{YAML} anatomy}

    \begin{itemize}\addtolength{\itemsep}{0.5\baselineskip}
        \item{Object description or list of object descriptions}
        \item{Instantiate an object with
              \textbf{!obj:<package>[.<subpackage>]*.<module>.<object>}}
        \item{Constructor arguments specified with \\
              \textbf{\{ <name>: <value>, ..., <name>: <value>\}}}
        \item{Objects are instantiated recursively}
        \item{Set an anchor (reference) to an object with
              \textbf{\&<anchorname> !obj: ...}}
        \item{Refer to an anchor with \textbf{*<anchorname>}}
        \item{For more details, see \url{http://deeplearning.net/software/pylearn2/yaml_tutorial/index.html}}
    \end{itemize}

\end{frame}

%------------------------------------------------------------------------------
% PYLEARN2 OVERVIEW
%------------------------------------------------------------------------------
\begin{frame}
    \frametitle{Pylearn2 overview}

    \begin{itemize}\addtolength{\itemsep}{0.5\baselineskip}
        \item{
            Train
            \begin{itemize}\addtolength{\itemsep}{0.5\baselineskip}
                \item{Dataset}
                \item{Model}
                \item{
                    TrainingAlgorithm
                    \begin{itemize}\addtolength{\itemsep}{0.5\baselineskip}
                        \item{Monitor}
                        \item{Cost}
                        \item{TerminationCriterion}
                    \end{itemize}
                }
                \item{TrainingExtension}
            \end{itemize}
        }
        \item{utils}
        \item{scripts}
    \end{itemize}

\end{frame}

%------------------------------------------------------------------------------
% TRAIN OBJECT
%------------------------------------------------------------------------------
\begin{frame}
    \frametitle{Pylearn2 overview: \textbf{Train} object}
    \Large
    \begin{itemize}\addtolength{\itemsep}{1.0\baselineskip}
        \item{Drives the main training loop}
        \item{
            Responsible for
            \begin{itemize}\addtolength{\itemsep}{0.5\baselineskip}
                \large
                \item{Starting training}
                \item{Stopping training}
                \item{Putting together the training algorithm, the model and the
                      dataset}
                \item{Managing misc. tasks before and after each training epoch}
                \item{Saving the trained model}
            \end{itemize}
        }
    \end{itemize}

\end{frame}

%------------------------------------------------------------------------------
% TRAININGALGORITHM OBJECT
%------------------------------------------------------------------------------
\begin{frame}
    \frametitle{Pylearn2 overview: \textbf{TrainingAlgorithm} object}
    \Large
    \begin{itemize}\addtolength{\itemsep}{1.0\baselineskip}
        \item{Drives the epoch training loop}
        \item{
            Responsible for
            \begin{itemize}\addtolength{\itemsep}{0.5\baselineskip}
                \large
                \item{Setting up the model}
                \item{Setting up the monitor}
                \item{Compiling the Theano function for parameter updates}
                \item{Doing one epoch's worth of parameter updates}
                \item{Save information about a training epoch via the monitor}
            \end{itemize}
        }
    \end{itemize}

\end{frame}

%------------------------------------------------------------------------------
% MODEL OBJECT
%------------------------------------------------------------------------------
\begin{frame}
    \frametitle{Pylearn2 overview: \textbf{Model} object}
    \Large
    \begin{itemize}\addtolength{\itemsep}{1.0\baselineskip}
        \item{Represents the mathematical model you want to optimize}
        \item{
            Responsible for
            \begin{itemize}\addtolength{\itemsep}{0.5\baselineskip}
                \large
                \item{Implementing the mapping from input to output that's
                      described by the mathematical model}
                \item{Describing the format of the data it expects to receive}
                \item{Storing the model's parameters}
            \end{itemize}
        }
        \item{There are multiple model frameworks (e.g. \textbf{MLP} and
              \textbf{DBM}, each is specialized in a different way)}
    \end{itemize}

\end{frame}

%------------------------------------------------------------------------------
% DATASET OBJECT
%------------------------------------------------------------------------------
\begin{frame}
    \frametitle{Pylearn2 overview: \textbf{Dataset} object}
    \Large
    \begin{itemize}\addtolength{\itemsep}{1.0\baselineskip}
        \item{Wraps around the dataset on which you train}
        \item{Common interface for all data}
        \item{
            Responsible for
            \begin{itemize}\addtolength{\itemsep}{0.5\baselineskip}
                \large
                \item{Storing the data}
                \item{Describing the format of the data it stores}
                \item{Instantiating iterators to loop over the data}
            \end{itemize}
        }
        \item{Main subclasses are \textbf{DenseDesignMatrix} and
              \textbf{SparseDataset}}
    \end{itemize}

\end{frame}

%------------------------------------------------------------------------------
% COST OBJECT
%------------------------------------------------------------------------------
\begin{frame}
    \frametitle{Pylearn2 overview: \textbf{Cost} object}
    \Large
    \begin{itemize}\addtolength{\itemsep}{1.0\baselineskip}
        \item{Represents a performance metric you want to maximize for the model}
        \item{
            Responsible for
            \begin{itemize}
                \item{Mapping the input to the cost expression as a Theano
                      expression}
                \item{Mapping the input to the cost gradient as a Theano
                      expression}
                \item{Describing the format of the input data it expects}
                \item{Describing cost-related quantities that are to be
                      monitored during training}
            \end{itemize}
        }
        \item{Possible to combine multiple costs using \textbf{SumOfCosts}}
    \end{itemize}

\end{frame}

%------------------------------------------------------------------------------
% MONITOR OBJECT
%------------------------------------------------------------------------------
\begin{frame}
    \frametitle{Pylearn2 overview: \textbf{Monitor} object}
    \begin{itemize}\addtolength{\itemsep}{0.5\baselineskip}
        \item{Holds information relative to training}
        \item{
            Responsible for
            \begin{itemize}\addtolength{\itemsep}{0.5\baselineskip}
                \large
                \item{Aggregating monitored quantities during training}
                \item{Compiling Theano function mapping input data to monitored
                      quantities}
            \end{itemize}
        }
    \item{Monitored quantities are called \emph{channels} and are implemented in
          the \textbf{MonitoringChannel} class}
    \item{Can monitor over multiple datasets (e.g. training, validation and test
          sets)}
    \end{itemize}

\end{frame}

%------------------------------------------------------------------------------
% TERMINATIONCRITERION OBJECT
%------------------------------------------------------------------------------
\begin{frame}
    \frametitle{Pylearn2 overview: \textbf{TerminationCriterion} object}
    \Large
    \begin{itemize}\addtolength{\itemsep}{2.5\baselineskip}
        \item{Determines when training has to stop}
        \item{Gets called between each training epoch}
    \end{itemize}

\end{frame}

%------------------------------------------------------------------------------
% TRAINEXTENSION OBJECT
%------------------------------------------------------------------------------
\begin{frame}
    \frametitle{Pylearn2 overview: \textbf{TrainExtension} object}
    \Large
    \begin{itemize}\addtolength{\itemsep}{1.0\baselineskip}
        \item{Represents a misc. task to be performed during training}
        \item{Gets called through \textbf{on\_monitor} (after the monitor has
              been called), \textbf{on\_save} (after the model has been saved)
              and \textbf{on\_setup} (right after the model has been
              instantiated)}
        \item{Use case: do early stopping (see \textbf{MonitorBasedSaveBest})}
    \end{itemize}

\end{frame}

%------------------------------------------------------------------------------
% UTILS MODULE
%------------------------------------------------------------------------------
\begin{frame}
    \frametitle{Pylearn2 overview: \textbf{utils} module}
    \begin{itemize}\addtolength{\itemsep}{0.5\baselineskip}
        \item{
            Lots of convenience functions: see
            \begin{itemize} 
                \item{\textbf{utils.sharedX}}
                \item{\textbf{utils.safe\_update}}
                \item{\textbf{utils.safe\_\{,i\}zip}}
                \item{\textbf{utils.safe\_update}}
                \item{\textbf{utils.function}}
                \item{\textbf{utils.grad}}
            \end{itemize} 
        }
        \item{
            \textbf{utils.serial}: meet your new best friend
            \begin{itemize}
                \item{\textbf{serial.load}: handles pretty much everything
                      related to loading files in various formats}
                \item{\textbf{serial.save}: handles pretty much everything
                      related to saving files in various formats}
                \item{Other serialization convenience functions are available,
                      you are encouraged to check them out on your own}
            \end{itemize}
        }
    \end{itemize}

\end{frame}

%------------------------------------------------------------------------------
% SCRIPTS DIRECTORY
%------------------------------------------------------------------------------
\begin{frame}
    \frametitle{Pylearn2 overview: \textbf{scripts} directory}
    \Large
    \begin{itemize}\addtolength{\itemsep}{0.5\baselineskip}
        \item{\textbf{plot\_monitor.py}: interactively lets you plot channels of
              a trained model's monitor}
        \item{\textbf{print\_monitor.py}: show all channel values of a trained
              model's monitor after training}
        \item{\textbf{show\_weights.py}: visually show a model's weights}
        \item{Once again, you are encouraged to explore the \textbf{scripts}
              directory on your own, lots of useful scripts are stored there}
    \end{itemize}

\end{frame}

%------------------------------------------------------------------------------
% EXTENDING
%------------------------------------------------------------------------------
\begin{frame}
    \frametitle{Extending the library}
    \LARGE
    Pylearn2 doesn't do what you want?
    \begin{itemize}\addtolength{\itemsep}{0.5\baselineskip}
            \large
            \item{Look at the \textbf{pylearn-users} mailing list
                  (\url{https://groups.google.com/d/forum/pylearn-users}),
                  the question might have been asked before}
            \item{If nothing answers your question, ask it; there probably is
                  something implemented but well-hidden}
            \item{If nothing suits your needs, most of the time subclassing one
                  element of the Pylearn2 library and overriding a few methods
                  is sufficient}
    \end{itemize}

\end{frame}

\end{document}
